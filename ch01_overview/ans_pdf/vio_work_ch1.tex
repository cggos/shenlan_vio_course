\documentclass[12pt,a4paper]{article}

\usepackage{xeCJK}
\usepackage{amsmath,amsthm,amssymb}
\usepackage{hyperref}
\usepackage{graphicx}
\usepackage{float}
\usepackage{makeidx}
\usepackage[numbers,sort&compress]{natbib}
\usepackage[level]{datetime}
\usepackage[top=1.5cm, bottom=2cm, outer=1.5cm, inner=2cm, heightrounded, marginparwidth=0cm, marginparsep=0cm]{geometry}
%\usepackage{showframe}

\renewcommand{\today}{\number\year 年 \number\month 月 \number\day 日}
\renewcommand{\refname}{参考文献}
\renewcommand{\tablename}{表}

\newcommand{\upcite}[1]{\textsuperscript{\textsuperscript{\cite{#1}}}}
\newcommand{\tabincell}[2]{\begin{tabular}{@{}#1@{}}#2\end{tabular}}

\title{第一节课习题}
\author{高洪臣}
\date{2019年6月14日}

\begin{document}

\maketitle

\section{VIO 文献阅读} 

\begin{enumerate}

\item 视觉与IMU进行融合之后有何优势?

\begin{table}[ht]
\centering
\caption{IMU vs 视觉}
\label{tab:imu_visual}
\begin{tabular}{|c|c|c|}
  \hline
  方案 & IMU & 视觉 \\ 
  \hline
  优势 &  
  \tabincell{c}{快速响应\\不受成像质量影响\\角速度普遍比较准确\\可估计绝对尺度} & 
  \tabincell{c}{不产生漂移\\直接测量旋转和平移} \\ 
  \hline
  劣势 & 
  \tabincell{c}{存在零偏\\低精度IMU积分位姿发散\\高精度价格昂贵} & 
  \tabincell{c}{受图像遮挡、运动物体干扰\\单目视觉无法测量尺度\\单目纯旋转运动无法估计\\快速运动时易丢失} \\ 
  \hline
\end{tabular}
\end{table}

根据表\ref{tab:imu_visual},视觉与IMU进行融合之后,可利用视觉定位信息来估计IMU的零偏,减少IMU由零偏导致的发散和累计误差;反之,IMU可以为视觉提供快速运动时的定位。


\item 有哪些常见的视觉+IMU融合方案?有没有工业界应用的例子?

(1)视觉+IMU融合方案 \cite{robotics7030045}
 \begin{itemize}
 \item 基于滤波方法:紧耦合(msckf和rovio)、松耦合(ssf)
 \item 基于优化方法:紧耦合(okvis和vins-mono)、松耦合
 \end{itemize}
 
(2)工业界应用例子
 \begin{itemize}
 \item 大疆无人机;
 \item 百度Apollo无人车;
 \item AR设备;
 \end{itemize}

\item 在学术界,VIO研究有哪些新进展?有没有将学习方法用到VIO中的例子?

(1)VIO研究新进展
 \begin{itemize}
 \item Visual-Inertial Mapping with Non-Linear Factor Recovery
 \item Stereo Visual Inertial LiDAR Simultaneous Localization and Mapping
 \end{itemize}

(2)将学习方法用到VIO中的例子
 \begin{itemize}
 \item Unsupervised Deep Visual-Inertial Odometry with Online Error Correction for RGB-D Imagery
 \item Selective Sensor Fusion for Neural Visual-Inertial Odometry
 \item Visual-Inertial Odometry for Unmanned Aerial Vehicle using Deep Learning
 \item Learning by Inertia: Self-supervised Monocular Visual Odometry for Road Vehicles
 \end{itemize}

\end{enumerate}



\section{四元数和李代数更新}

编程验证对于小量 $\boldsymbol{\omega} = [0.01, 0.02, 0.03]^T$,两种对旋转更新的方式得到的结果非常接近。

\begin{equation}
\mathbf{R} \leftarrow \mathbf{R} \exp(\boldsymbol{\omega}^{\wedge})
\end{equation}

\begin{equation}
\mathbf{q} \leftarrow \mathbf{q} \otimes [1, \frac{1}{2} \boldsymbol{\omega}]^T
\end{equation}

\noindent
程序见 \textbf{code} 目录。



\section{其他导数}

使用右乘 $\mathfrak{so}(3)$,推导以下导数 \cite{Gao2017SLAM}

\begin{equation}
\begin{aligned}
\frac{d(\mathbf{R}^{-1} \mathbf{p})}{d \mathbf{R}} 
&= \frac{\partial{(\mathbf{R}^{-1} \mathbf{p})}}{\partial{\varphi}} \\
&= \lim_{\varphi \rightarrow 0} 
   \frac{(\mathbf{R}\exp(\varphi^{\wedge}))^{-1} \mathbf{p} - \mathbf{R}^{-1} \mathbf{p}} {\varphi} \\
&= \lim_{\varphi \rightarrow 0} 
   \frac{\exp(-\varphi^{\wedge}) \mathbf{R}^{-1} \mathbf{p} - \mathbf{R}^{-1} \mathbf{p}} {\varphi} \\
&\approx 
   \lim_{\varphi \rightarrow 0} 
   \frac{(\mathbf{I}-\varphi^{\wedge}) \mathbf{R}^{-1} \mathbf{p} - \mathbf{R}^{-1} \mathbf{p}} {\varphi} \\
&= \lim_{\varphi \rightarrow 0} 
   \frac{-\varphi^{\wedge} \mathbf{R}^{-1} \mathbf{p}} {\varphi} \\
&= \lim_{\varphi \rightarrow 0} 
   \frac{(\mathbf{R}^{-1}\mathbf{p})^{\wedge} \varphi} {\varphi} \\
&= (\mathbf{R}^{-1}\mathbf{p})^{\wedge}
\end{aligned}
\end{equation}


\begin{equation}
\begin{aligned}
\frac{d \ln(\mathbf{R}_1 \mathbf{R}_2^{-1})^{\vee}}{d \mathbf{R}_2}
&= \lim_{\phi \rightarrow 0} 
   \frac{\ln(\mathbf{R}_1 (\mathbf{R}_2 \exp(\phi^{\wedge}))^{-1})^{\vee} - \ln(\mathbf{R}_1 \mathbf{R}_2^{-1})^{\vee}} {\phi} \\
&= \lim_{\phi \rightarrow 0} 
   \frac{\ln(\mathbf{R}_1 \exp(-\phi^{\wedge}) \mathbf{R}_2^{-1})^{\vee}  - \ln(\mathbf{R}_1 \mathbf{R}_2^{-1})^{\vee}} {\phi} \\
&= \lim_{\phi \rightarrow 0} 
   \frac{\ln(\mathbf{R}_1 \mathbf{R}_2^{-1} \exp((-\mathbf{R}_2 \phi)^{\wedge}))^{\vee} - \ln(\mathbf{R}_1\mathbf{R}_2^{-1})^{\vee}} {\phi} \\
&= \lim_{\phi \rightarrow 0} 
   \frac{\ln(\mathbf{R}_1\mathbf{R}_2^{-1})^{\vee} + \mathbf{J}_r^{-1} (-\mathbf{R}_2 \phi) - \ln(\mathbf{R}_1\mathbf{R}_2^{-1})^{\vee}} {\phi} \\
&= -\mathbf{J}_r^{-1}(\ln(\mathbf{R}_1\mathbf{R}_2^{-1})^{\vee}) \mathbf{R}_2
\end{aligned}
\end{equation}


\bibliographystyle{unsrt}
\bibliography{bibfile}

\end{document}

